\documentclass[12pt,letterpaper]{article}

\usepackage[utf8]{inputenc}
\usepackage[spanish]{babel}
\usepackage{times}
\usepackage[left=3cm,top=2.5cm,bottom=2.5cm,right=2.5cm]{geometry}
\usepackage{graphicx}
\title{EV\_ 2\_ 3\_ Explicar\_ los\_ arreglos\_ y\_ parámetros\_ de\_ los\_ Amplificadores\_ Clase\_ B.}


\begin{document}
\maketitle




\paragraph{ UNIVERSIDAD POLITÉCNICA DE LA ZONA METROPOLITANA DE GUADALAJARA}

\
\begin{figure}[h!]
\begin{center}

\includegraphics[scale=0.8]{Upzmg.png} 
\label{Upzmg}


\end{center}
\end{figure}


\

\large{Perez de Alba Santiago Eduardo.\\
Fecha: 07 de Octubre del 2019.
\

Curso: Sep-Nov 2019.

\
Carrera: Ingeniería en Mecatronica.\

Docente: Moran Garabito Carlos Enrique}

\newpage

\section{Marco teórico:}
\

El amplificador clase B se caracteriza porque cada transistor se encarga de amplificar la señal en la mitad de su periodo (180 grados).
Uno de los principales usos de los amplificadores clase B son, para mejorar el rendimiento y aprovechar al máximo la potencia entregada por la fuente de alimentación para la conversión.
\

Como norma general, los amplificadores clase B, se montan con transistores que trabajen en contrafase (push-pull); con el fin de minimizar los armónicos que se pueden llegar a generar en los montajes.

Estos funcionan cuando la tension de polarización y la amplitud máxima de la señal de entrada poseen valores tales que hacen que la corriente de salida circule durante un semiperíodo de la señal de entrada.
En los amplificadores de clase B, puede que no haya nunca corriente de reja (base).

\

Los amplificadores Clase B utilizan dos transistores complementario o coincidentes, uno de tipo NPN y otro del tipo PNP con ambos transistores de potencia que reciben la misma señal de entrada que es igual en magnitud, pero en fase opuesta entre sí. Esto resulta en un transistor solamente amplifica la mitad o 180 grados o del ciclo de forma de onda de entrada mientras que el otro transistor amplifica la otra mitad o restante 180 grados con las resultantes de "dos mitades" esta poniendo juntos de nuevo en la salida terminal.

\

El funcionamiento del amplificador clase B tiene polarización CC cero ya que los transistores están polarizados en el corte, por lo que cada transistor solo conduce cuando la señal de entrada es mayor que la tension del emisor base. Por lo tanto, en la entrada cero hay salida cero y no se consume energía. Esto significa que el punto Q real de un amplificador clase B esta en la parte Vce de la linea de carga.

\



\end{document}