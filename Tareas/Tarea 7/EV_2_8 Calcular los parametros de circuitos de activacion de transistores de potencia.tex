\documentclass[14pt,letterpaper]{article}
\usepackage[utf8]{inputenc}
\usepackage[spanish]{babel}
\usepackage{times}
\usepackage[left=3cm,top=2.5cm,bottom=2.5cm,right=2.5cm]{geometry}
\usepackage{graphicx}
\author{Perez de Alba Santiago Eduardo.}

\begin{document}
\begin{center}

\LARGE \textbf{Universidad Politecnica de la Zona Metropoilitana de Guadalajara\\}

\includegraphics[scale=1]{Upzmg.png} 

\large \textbf{EV\_ 2\_ 8 Calcular los parametros de circuitos de activación de transistores de potencia.}

\end{center}

\vspace{1cm}
\large \textbf{Nombre: Perez de Alba Santiago Eduardo.\\
\vspace{0.5cm} Carrera: Ingeniería en Mecatronica.\\
\vspace{0.5cm} Materia: Sistemas Electrónicos de Interfaz.\\
\vspace{0.5cm} Curso: Septiembre-Noviembre del 2019.\\
\vspace{0.5cm} Docente: Moran Garabito Carlos Enrique.}\\
\vspace{0.5cm}
\small \textbf{29 de Octubre del 2019}


\newpage
\section{Transistor de potencia como Interruptor:}
El funcionamiento y utilización de los transistores de potencia es idéntico al de los transistores normales, teniendo como características especiales las altas tensiones e intensidades que tiene que soportar las altas potencias a disipar.

\
Existen tres tipos de transistores de potencia:
\begin{itemize}
\item Bipolar
\item Unipolar o FET(Transistor de Efecto de Campo)
\item IGBT
\end{itemize}

\
Estos tipos de transistores pueden operar como un interruptor mecánico, con limitaciones respecto a un conmutador ideal. Estas limitaciones, restricciones para algunas aplicaciones por lo que debemos conocer las características y especificaciones de estos dispositivos para su adecuación al uso que se le va a dar.

\
Un transistor se usa para cambiar la operación de apertura o cierre de un circuito. Este tipo de conmutación de estado solido ofrece una fiabilidad significativa. Los transistores NPN y PNP se pueden usar como conmutadores.

\
\section{Cálculos:}
\
\subsection{TIP41C:}
\begin{itemize}
\item Transistor NPN
\item $I_c$ max: 6A
\item $I_c$ pico max: 10A ($t_p$<5ms)
\item $I_B$ max: 3A
\item $P_{TOT}$: 65W
\item $V_{CEO}$: 100V, $V_{CBO}$: 100V, $V_{EBO}$: 5V
\item $h_{FE}$: 15 a 75 (@$I_C$=3A, $V_{CE}$=4V)
\item Alta velocidad de switcheo
\item Encapsulado: TO-220

\end{itemize}

\

$$R=\frac{V}{Ib}$$
\

$$Ib=\frac{Ic}{h_{FE}(min)}$$

\
$$Ib=\frac{6A}{15}=0.4Ib$$
\
$$12-0.7=11.3V$$

$$R=\frac{11.3}{0.4}=285.25$$
\
\subsection{Ejemplo:}
\begin{itemize}
\item Rb=30kohm
\item Rc=0.7k ohm
\item $V_{CC}$= 12V
\item $\beta$ =125
\item $V_i$=0v y 5v

\end{itemize}

\
cuando VCE= 0
$$I_c= \frac{V_{CC}}{R_C}$$
$$I_c=\frac{12V}{0.7kohm}$$
$$I_c= 17.1mA$$

\
$$Ib= \frac{Ic}{\beta}$$
$$Ib=\frac{17.1mA}{125}$$
$$Ib=0.1368A$$

\
Considerando que el voltaje de entrada aplicado es de 12V, la corriente base puede determinarse aplicando la ley de voltaje de Kirchhoff.

\
$$Ib=\frac{Vi-Vbe}{Rb}$$
Para el transistor de silicio $Vbe=0.7v$\\ Por lo tanto, $Ib=\frac{12V-0.7V}{30kohm}= 0.000376A$
\
Por lo tanto la salida del colector llega a ser aproximadamente cero.
\end{document}